\documentclass[12pt]{article}

% --- PACKAGES ---
\usepackage{amsmath, amssymb}
\usepackage[margin=1in]{geometry}
\usepackage{enumitem} % Added for custom list labels

% --- TITLE INFORMATION ---
\title{\textbf{Building the Genomic Relationship Matrix}}
\author{Fei Ge}
\date{\today}

% ==============================================================================
\begin{document}

\maketitle

\noindent This is the foundation. It quantifies the genetic similarity
between individuals based on genome-wide markers.

\vspace{1em}

\subsection*{Step 1: Create and Center the Genotype Matrix}

\textbf{Input:} A raw genotype matrix $W$ ($n \times m$) of $n$ individuals and $m$ markers,
coded 0, 1, and 2.

\bigskip

\begin{enumerate}
    \item \textbf{Calculate allele frequencies ($p_i$):}
    For each marker $i$, calculate the allele frequency
    \[
    p_i = \frac{\text{total count of allele at marker } i}{2 \times (\text{number of individuals})}.
    \]

    \item \textbf{Create the Frequency Matrix ($P$):}
    Construct a matrix $P$ where every element in column $i$ is the value $2p_i$.
    This represents the expected genotype score based on allele frequency.

    \item \textbf{Center the Genotype Matrix ($M$):}
    Subtract $P$ from the raw genotype matrix $W$:
    \[
    M = W - P.
    \]
    This adjustment centers the genotypes by allele frequencies, giving each marker a mean of zero.
\end{enumerate}

\textbf{Example:} Suppose we have 4 individuals and 3 markers.
Then the raw genotype matrix $W$ is

\[
W =
\begin{bmatrix}
1 & 2 & 0 \\
1 & 1 & 1 \\
2 & 2 & 0 \\
0 & 1 & 2
\end{bmatrix}.
\]

\begin{enumerate}
    \item \textbf{Calculate the allele frequencies ($p_i$):}
    The total number of alleles is $2 \times 4 = 8$. For each marker (column):
    \[
    p_1 = \tfrac{4}{8} = 0.5, \quad
    p_2 = \tfrac{6}{8} = 0.75, \quad
    p_3 = \tfrac{3}{8} = 0.375.
    \]

    \item \textbf{Create the frequency matrix ($P$):}
    Each column $i$ contains the value $2p_i$ (Since each individual has two chromosomes, their expected genotype score is not $p_i$, but $2p_i$):
    \[
    2p_1 = 1.0, \quad 2p_2 = 1.5, \quad 2p_3 = 0.75.
    \]
    So
    \[
    P =
    \begin{bmatrix}
    1.0 & 1.5 & 0.75 \\
    1.0 & 1.5 & 0.75 \\
    1.0 & 1.5 & 0.75 \\
    1.0 & 1.5 & 0.75
    \end{bmatrix}.
    \]

    \item \textbf{Center the matrix ($M = W - P$):}
    Subtract $P$ from $W$:
    \[
    M =
    \begin{bmatrix}
    1 & 2 & 0 \\
    1 & 1 & 1 \\
    2 & 2 & 0 \\
    0 & 1 & 2
    \end{bmatrix}
    -
    \begin{bmatrix}
    1.0 & 1.5 & 0.75 \\
    1.0 & 1.5 & 0.75 \\
    1.0 & 1.5 & 0.75 \\
    1.0 & 1.5 & 0.75
    \end{bmatrix}
    =
    \begin{bmatrix}
    0 & 0.5 & -0.75 \\
    0 & -0.5 & 0.25 \\
    1.0 & 0.5 & -0.75 \\
    -1.0 & -0.5 & 1.25
    \end{bmatrix}.
    \]
\end{enumerate}
\subsection*{Step 2: Scale the Centered Matrix and Calculate $G$ }

\begin{enumerate}
    \item \textbf{Calculate the Scaling Factor:}
    The denominator used for scaling is the sum of the variances of all the markers. This is calculated as:
    \[
    k = 2 \sum_{i=1}^{m} p_{i}(1 - p_{i})
    \]
    where $m$ is the total number of markers. This factor ensures that the average diagonal element of the final $G$ matrix is close to 1, making it analogous to a traditional pedigree-based relationship matrix.

    \textbf{Explanation:} The scaling factor $k$ is the sum of the expected variances of all markers. The term $2p_i(1-p_i)$ comes from the statistical variance of a single genetic marker, assuming Hardy-Weinberg equilibrium.

    \begin{itemize}
        \item \textbf{Single Allele as a Bernoulli Trial:}
        Consider drawing one allele from the population. It is either the allele of interest (value 1) or the other allele (value 0). The variance is
        \[
        \text{Var(allele)} = p_i(1-p_i)
        \]

        \item \textbf{Genotype as Sum of Two Alleles:}
        An individual's genotype $W$ is the sum of two alleles (one from each parent). Assuming independence (Hardy-Weinberg Equilibrium):
        \[
        W = \text{Allele}_1 + \text{Allele}_2
        \]

        \item \textbf{Variance of the Genotype:}
        Using the property that the variance of the sum of independent variables is the sum of their variances:
        \[
        \text{Var}(W) = 2 p_i(1-p_i)
        \]

        \item \textbf{Scaling Factor $k$:}
        Summing across all markers gives
        \[
        k = \sum_{i=1}^{m} \text{Var}(W_i) = 2 \sum_{i=1}^{m} p_i(1-p_i)
        \]
    \end{itemize}

    \item \textbf{Calculate the Genomic Relationship Matrix $G$:}
    Once you have $M$ and $k$, compute $G$ as:
    \[
    G = \frac{M M'}{k}
    \]
    \begin{itemize}
        \item $M$: the centered genotype matrix from Step 1.
        \item $M'$: the transpose of $M$.
        \item $M M'$: matrix multiplication that computes pairwise genomic relationships.
        \item $k$: the scaling factor calculated above.
    \end{itemize}

    \textbf{Explanation: Dot Product = Similarity}

    The reason we multiply $M$ by its transpose $M'$ comes from linear algebra and statistics:

    \subsubsection*{Dot Product: A Measure of Similarity}
    Each row of $M$ is a vector representing an individual's deviation from population allele frequencies across all markers.
    - A large positive dot product between two rows indicates similar deviations (genetic similarity).
    - A dot product near zero indicates no correlation.
    - A large negative dot product indicates opposite deviations (genetic dissimilarity).

    \subsubsection*{Link to Covariance}
    The covariance between two vectors $X$ and $Y$ is
    \[
    \text{Cov}(X,Y) = \sum_i (X_i - \bar{X})(Y_i - \bar{Y})
    \]
    Since $M$ is mean-centered, the dot product of rows $i$ and $j$ is exactly the covariance between individuals $i$ and $j$.

    \subsubsection*{Gram Matrix Interpretation}
    - $MM'$ produces an $n \times n$ matrix of dot products between individuals (rows).
    - The diagonal elements $(MM')_{ii}$ measure an individual's total variance.
    - The off-diagonal elements $(MM')_{ij}$ measure the genetic covariance between individuals.

    In other words, $MM'/k$ is a scaled \textbf{Gram matrix}, giving the genomic relationship matrix used in GBLUP.
\end{enumerate}
\subsubsection*{Example: Continue from Step 1}

\textbf{Step 2: Calculate the $G$ Matrix}

\begin{enumerate}[label=\alph*)]
    \item \textbf{Calculate the Numerator ($MM'$):}
    Multiply the centered genotype matrix $M$ by its transpose:
    \[
    M =
    \begin{bmatrix}
    0 & 0.5 & -0.75 \\
    0 & -0.5 & 0.25 \\
    1 & 0.5 & -0.75 \\
    -1 & -0.5 & 1.25
    \end{bmatrix}, \quad
    M' =
    \begin{bmatrix}
    0 & 0 & 1 & -1 \\
    0.5 & -0.5 & 0.5 & -0.5 \\
    -0.75 & 0.25 & -0.75 & 1.25
    \end{bmatrix}
    \]

    \[
    MM' =
    \begin{bmatrix}
    0.8125 & -0.4375 & 0.8125 & -1.1875 \\
    -0.4375 & 0.3125 & -0.4375 & 0.5625 \\
    0.8125 & -0.4375 & 1.8125 & -2.1875 \\
    -1.1875 & 0.5625 & -2.1875 & 2.8125
    \end{bmatrix}
    \]

    \item \textbf{Calculate the Scaling Factor $k$:}
    Using the formula
    \[
    k = 2 \sum_{i=1}^{m} p_i(1-p_i),
    \]
    we compute the variance component for each marker:

    \[
    \begin{aligned}
    \text{Marker 1: } & 2 \times 0.5 \times (1 - 0.5) = 0.5 \\
    \text{Marker 2: } & 2 \times 0.75 \times (1 - 0.75) = 0.375 \\
    \text{Marker 3: } & 2 \times 0.375 \times (1 - 0.375) = 0.46875
    \end{aligned}
    \]

    Summing these gives the total scaling factor:
    \[
    k = 0.5 + 0.375 + 0.46875 = 1.34375
    \]

    \item \textbf{Calculate the Final $G$ Matrix:}
    Divide each element of $MM'$ by the scaling factor $k = 1.34375$:
    \[
    G = \frac{MM'}{k} \approx
    \begin{bmatrix}
    0.605 & -0.326 & 0.605 & -0.884 \\
    -0.326 & 0.233 & -0.326 & 0.419 \\
    0.605 & -0.326 & 1.349 & -1.628 \\
    -0.884 & 0.419 & -1.628 & 2.093
    \end{bmatrix}
    \]
\end{enumerate}

\subsubsection*{Interpreting the $G$ Matrix}

\begin{itemize}
    \item \textbf{Diagonal Elements:}
    Represent an individual's relationship with itself. Values greater than 1 (e.g., Ind3 and Ind4) indicate more homozygosity than average.

    \item \textbf{Off-Diagonal Elements:}
    Represent the estimated genomic relationship between two individuals:
    \begin{itemize}
        \item Ind1 and Ind3: 0.605 → genetically similar (they share similar alleles).
        \item Ind1 and Ind4: -0.884 → genetically dissimilar (opposite deviations from average).
    \end{itemize}
\end{itemize}

\end{document}
